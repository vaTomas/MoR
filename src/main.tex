\documentclass[journal, a4paper]{IEEEtran}

\usepackage[hidelinks]{hyperref}  % PDF Hyperlinks


\usepackage{lipsum}               % Lorem Ipsum





% correct bad hyphenation here
\hyphenation{op-tical net-works semi-conduc-tor}


\begin{document}
\title{A Literature Review on NDT and SHM}


% === Authors === %
\author{Antonette~C.~Maxey,~\IEEEmembership{CPE Student,~MMCM,}
        Florencio~N.~Pulido,~\IEEEmembership{CPE Student,~MMCM,}
        and~Vincent~Alfred~B.~Tomas,~\IEEEmembership{ECE Student,~MMCM}% <-this % stops a space
}



% The paper headers
\markboth{Methods of Research Literature Review, February~2024}%
{Shell \MakeLowercase{\textit{et al.}}: Developing a Portable NDT Device for Efficient SHM}


% === Build Title Area  === %
\maketitle


\begin{abstract}
The abstract goes here. \lipsum[1]
\end{abstract}


% Note that keywords are not normally used for peerreview papers.
\begin{IEEEkeywords}
  NDT, SHM, Ultrasonic, Concrete.
\end{IEEEkeywords}







\section{Introduction}
\IEEEPARstart{S}{tructural} Health Monitoring (SHM) is a pivotal engineering tool, providing real-time structural integrity data,
thereby ensuring safety and averting infrastructure failures \cite{Gharehbaghi2022} \cite{Katam2023}.
Traditional SHM methods, while effective, are often limited by factors such as
high cost, complexity, and accessibility issues \cite{Katam2023} \cite{Gharehbaghi2022}.
A portable, non-destructive testing (NDT) device presents a promising solution for efficient and accessible SHM,
overcoming the limitations of traditional methods \cite{Guo2022} \cite{Chen2023}.
The objective of this capstone project is to explore the potential of a portable,
non-destructive testing (NDT) device in overcoming the limitations of traditional Structural Health Monitoring (SHM) methods.


\section{Background and Theoretical Framework}
\begin{itemize}
  \item Define key terms: SHM, NDT, portable NDT device.
  \item Discuss the different types of NDT methods and their advantages and disadvantages.
  \item Review existing portable NDT devices and their functionalities and limitations.
  \item Briefly discuss relevant theories and frameworks related to structural safety, damage detection, and sensor technology.
\end{itemize}


\section{Literature Review}

\subsection{Existing Portable NDT Devices}
\begin{itemize}
  \item Review recent research and development efforts in portable NDT devices for SHM applications.
  \item Analyze the functionalities, target applications, and performance characteristics of various devices.
  \item Discuss the limitations and challenges identified in existing research.
\end{itemize}

\subsubsection{Sensor Technology for SHM}
\begin{itemize}
  \item Review advancements in sensor technology relevant to portable NDT devices (e.g., miniaturization, low-power consumption, wireless communication).
  Discuss the integration of different sensor types and their data fusion approaches for comprehensive SHM.
  Analyze the impact of sensor technology limitations on device performance and reliability.
\end{itemize}

\subsection{Data Acquisition and Analysis for SHM}
\begin{itemize}
  \item Review methods for data acquisition, processing, and analysis from portable NDT devices.
  \item Discuss approaches for real-time data monitoring, damage detection algorithms, and data visualization techniques.
  \item Analyze the challenges of data management and interpretation for efficient SHM decision-making.
\end{itemize}

\subsection{Applications and Case Studies}
\begin{itemize}
  \item Review successful applications of portable NDT devices in various SHM scenarios (e.g., bridges, buildings, industrial structures).
  \item Analyze the cost-effectiveness, ease of use, and reliability of these devices in real-world settings.
  \item Discuss potential new applications and emerging trends in portable NDT technology for SHM.
\end{itemize}


\section{Synthesis and Critique}
\begin{itemize}
  \item Summarize the key findings from the literature review, highlighting advancements, limitations, and research gaps.
  \item Critically evaluate the effectiveness of existing portable NDT devices for efficient SHM.
  \item Identify opportunities for improvement and potential directions for future development.
\end{itemize}


\section{Conclusion}
\begin{itemize}
  \item Restate the research question or objective and summarize the main findings of the literature review.
  \item Discuss the implications of your findings for the development of your proposed portable NDT device.
  \item Outline the next steps in your capstone project, including design considerations, testing methods, and expected outcomes.
\end{itemize}








% Can use something like this to put references on a page
% by themselves when using endfloat and the captionsoff option.
\ifCLASSOPTIONcaptionsoff
  \newpage
\fi


% === Bibliography ===

\bibliographystyle{IEEEtran}
\bibliography{bibtex/bib/references}

% === End of Bibliography ===


\end{document}


